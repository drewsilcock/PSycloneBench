\documentclass[12pt]{article}
\usepackage[pdftex]{graphicx}
% and their extensions so you won't have to specify these with
% every instance of \includegraphics
\DeclareGraphicsExtensions{.pdf,.jpeg,.png}
\usepackage{url}

\newcommand{\psykal}{{PS}y{KA}l}

\begin{document}

\title{Analysing the Performance of Shallow-Water Models with the
 Roofline Model}

\author{A.~.R.~Porter and R.~W.~Ford}

\maketitle

\section{Performance Analysis with the Roofline Model}

Although we have investigated how the performance of the \psykal
version of nemolite2d compares with that or the original, we have not
addressed how efficient the original actually is. In order to do so we
use the Roofline Model~\cite{roofline} which provides a relatively
simple way of characterising the performance of a code in terms of
whether it is memory-bandwidth bound or compute bound. To do so we
follow the approach suggested in~\cite{para_pearls} and use the
well-known STREAM~\cite{stream} and LINPACK~\cite{linpack} benchmarks
in order to get the upper bounds on the memory bandwidth and FLOPS for
the test hardware. Since we are using an Intel CPUs we used
the Intel Math Kernel Library implementation of LINPACK.

A key component of the Roofline model is the Arithmetic Intensity of
the code being executed:
\begin{equation}
AI = \frac{\textrm{No. of floating-point operations}}{\textrm{Bytes fetched from memory}}
\end{equation}
We calculated this quantity manually by examining the
source code and counting the number of memory references and
arithmetic operations that it contained. In doing this counting we
assume that any references to adjacent array elements are fetched in a
single cache-line and thus only count once.

To illustrate this, consider the following code fragment (which forms
the body of a doubly-nested loop over {\tt i} and {\tt j} with {\tt i}
innermost):
\begin{verbatim}
CU(I+1,J) = .5d0*(P(I+1,J)+P(I,J))*U(I+1,J)
CV(I,J+1) = .5d0*(P(I,J+1)+P(I,J))*V(I,J+1)
Z(I+1,J+1) =(FSDX*(V(I+1,J+1)-V(I,J+1))- &
             FSDY*(U(I+1,J+1)-U(I+1,J)))/&
            (P(I,J)+P(I+1,J)+            &
             P(I+1,J+1)+P(I,J+1))
H(I,J) = P(I,J)+0.25d0*                   &
         (U(I+1,J)*U(I+1,J)+U(I,J)*U(I,J) & 
         +V(I,J+1)*V(I,J+1)+V(I,J)*V(I,J))
\end{verbatim}
This code writes to four memory locations ({\tt CU}, {\tt CV}, {\tt Z}
and {\tt H}) and reads from ten distinct locations.  However, if we
assume caching of reads from adjacent locations ({\it e.g.} {\tt
  P(i,j)} and {\tt P(i+1,j)}) then we have only six distinct reads.
If we assume that each of these ten memory accesses will touch a
separate cache line then this code fragment will result in the moving
of ten cache lines, each of which is 64 bytes (eight double-precision
words). However, as noted above, this code fragment is the body of a
loop nest. Therefore, upon the next loop trip {\tt i} will have been
incremented by one and the majority of the accessed values will still
be in cache. In fact, a new cache line will only have to be fetched
every seven trips of the (innermost) loop and thus the mean memory
traffic per loop trip is approximately $640/7$ bytes.

We can also count the number of floating-point arithmetic operations
in this code fragment which gives us 12 multiplications and 12
additions. $AI$ for this code fragment is then $24*7/640 = 0.26$
FLOPs/byte.  If we do the same analysis for the (longer) x-Momentum
kernel of nemolite2d then we find that there are 51 addition
operations, 56 multiplication/division operations and one call to
sin(). There are 31 distinct memory references. This gives an $AI$ of
approximately $108*7/(31*64)$ or 0.38 FLOPs/byte.

We measure the mean time to execute this region of code as 0.0020
seconds (Intel v.16 on Haswell E5-1620 v2) for the 256 domain. This
gives us $\frac{108*256^2}{0.002} = 3.5$ GFLOPS. In contrast,
instrumenting this section of code with the LIKWID marker API and
using likwid-perfctr we measure 4.7 GFLOPS.

For the 128 the measured time per kernel call is 0.570033E-03 s.
$108*128^2/0.570e-3 = 3.1$ GFLOPS which is 25\% less than we measure
directly with likwid.  We can use the value reported by likwid to get
the FLOP count: 4.269 GFLOPS $\times 1.677 s = 7.1615217$ GFLOPs. This is for
3000 steps of $128^2$ so one kernel call is: $7.1615217/(3000*128^2) =
145.7$ FLOPs. This means that $AI = 145.7*7/(31*64) = 0.514$ FLOP/byte

\begin{figure}
\centering
\includegraphics[width=120mm]{roofline_archer}
\caption{Comparison of the performance achieved by nemolite2d and Shallow on Archer (Intel Ivybridge)}
\label{FIG_roofline_archer}
\end{figure}

\begin{figure}
\centering
\includegraphics[width=120mm]{roofline_haswell}
\caption{Comparison of the performance achieved by nemolite2d and Shallow on Intel Haswell}
\label{FIG_roofline_haswell}
\end{figure}

\begin{table}
\begin{tabular}{l|l|l}
\hline
\hline
                                    &    128       &   256 \\
\hline
        Runtime (RDTSC) [s]        &   1.697142   &     6.446171  \\
        Runtime unhalted [s]       & 1.750448e+00 &   6.734944e+00 \\
            Clock [MHz]            & 3.818981e+03 &   3.875309e+03 \\
                CPI                & 6.503796e-01 &   6.348858e-01 \\
\hline
        L1 DTLB load misses        &    1317912   &    6748712   \\
       L1 DTLB load miss rate      & 1.326630e-04 & 1.723565e-04 \\
  L1 DTLB load miss duration [Cyc] & 2.101039e+01 & 2.013584e+01 \\
        L1 DTLB store misses       &     87257    &    401036    \\
      L1 DTLB store miss rate      & 8.783423e-06 & 1.024213e-05 \\
 L1 DTLB store miss duration [Cyc] & 2.151055e+01 & 2.669793e+01 \\
\hline
  L2D load bandwidth [MBytes/s] & 1.080936e+04 & 1.446214e+04 \\
  L2D load data volume [GBytes] &  18.1215488  &   92.177536  \\
 L2D evict bandwidth [MBytes/s] & 4.501487e+02 & 5.065637e+02 \\
 L2D evict data volume [GBytes] &  0.75465984  &  3.22869248  \\
     L2 bandwidth [MBytes/s]    & 1.130282e+04 & 1.498363e+04 \\
     L2 data volume [GBytes]    & 18.948807808 & 95.501414592 \\
\hline
  L3 load bandwidth [MBytes/s] & 4.511686e+03 & 4.607617e+03 \\
  L3 load data volume [GBytes] &   7.6168768  &   29.692832  \\
 L3 evict bandwidth [MBytes/s] & 2.771148e+02 & 2.722823e+02 \\
 L3 evict data volume [GBytes] &  0.467840512 &  1.75466688  \\
    L3 bandwidth [MBytes/s]    & 4.788801e+03 & 4.879899e+03 \\
    L3 data volume [GBytes]    &  8.084717312 &  31.44749888 \\
\hline
\hline
\end{tabular}
\end{table}

\bibliographystyle{unsrt}
\bibliography{gocean_roofline}

\end{document}
